% --------------- 12 POINT FONT -------------------------------
\documentclass[12pt]{article}
% --------------- 10 POINT FONT FOR CAPTIONS ------------------
\usepackage[font=footnotesize]{caption}
% --------------- New TIMES Roman FONT -------------------------------
\usepackage{times}
% --------------- 1 INCH MARGINS ------------------------------
\usepackage[margin=1in]{geometry}
% --------------- LINE SPACING --------------------------------
\usepackage{setspace}
\singlespacing
%\doublespacing
% --------------- SMALL SECTION TITLES ------------------------
\usepackage[tiny,compact]{titlesec}
\usepackage{graphicx}
\usepackage{hyperref}
\renewcommand*{\figureautorefname}{Fig.} 
\renewcommand{\figurename}{Fig.}
\usepackage{wrapfig}
\usepackage{setspace}
\usepackage[square,sort,comma,numbers]{natbib}
\setlength{\bibsep}{0pt}
\bibliographystyle{ieeetr}
\usepackage[
activate = {true},
kerning  = true,
spacing  = true,
factor   = 1000,
stretch  = 30,
shrink   = 30,
]{microtype}
\pagestyle{empty}

\titleformat{\section}[display]%
{\normalfont\bfseries}
{}{0pt}{}
\titlespacing*{\section}{0pt}{-1.3em}{0pt}

\begin{document}
\begin{center}
  {\large \textbf{Statement of Research Interests}} \\[0.1em]
  {Constantinos Chamzas}
\end{center}
\vspace{-10pt}

\section{}
My research interests have primarily focused on leveraging a robots' past experiences to imporve its motion planning performance in future similar problems. %
Specifically, I am interested in improving sampling-based planners    
Specifically I have focused on sampli computing conidtional sampling-d

leveraging a ro
Ideally, a robot's past \textit{experiences} should inform future actions in order to improve performance over time.
by (1) learning to exploit similarities between motion planning problems and (2) creating adaptive algorithms that use prior experience and are robust across a wide range of scenarios.



%However, motion planning, the field that designs the trajectories a robotic arm or car must follow, currently focuses on solving one problem at a time.
%This strategy works for easy problems -- but a robot that always plans from scratch, always forgetting, will never reach human-level performance in complex environments. \textit{The goal of this proposal is to transform how robots plan motions} 
%\section{Future interests}
%In motion planning, the advent of sampling-based planners has enabled researchers to compute paths for complex robots. These are algorithms that explore the high-dimensional solution space, by discovering free placements of the robot through sampling. 
%Recent approaches have demonstrated that using past experiences can substantially increase the performance of a planner; however, they only work in idealized situations. One set of approaches (e.g.,~\cite{Coleman2015})  stores complete solutions in the form of entire paths that are later retrieved and repaired. However, a  database of complete paths includes both redundant information as well as hard-coded paths that easily get invalidated by changes in the environment. Another set of techniques (e.g.,~\cite{Ichter2018}) uses learning methods to modify the sampling procedures of the planner but considers at once the entire environment in which the robot operates for learning these procedures. Subsequently, even minor changes in the environment, which may result in drastically different solution paths, cannot be captured or learned efficiently. The two main questions that prior work has failed to address are (1) how to transfer experiences between environments by successfully defining their similarity and (2) how to store and retrieve them efficiently. 
%
%\section{Hypothesis and Research Plan (Intellectual Merit)}
%In order to overcome these hurdles, I plan to investigate an experience-based framework
%that builds on prior work but targets the problem \textit{locally} rather than globally. My inspiration arises from the fact that the motion planning problem can become very challenging in narrow parts of the environment. 
%If a framework can capture this local information and combine it on demand, it could arrive at global plans that deal effectively and locally with difficult parts of the solution space.
%
%A proof of concept of this idea has already been demonstrated in my recently submitted work~\cite{Chamzas2019} and is illustrated on a planar manipulator shown in \autoref{fig:pipeline}. The local primitives are defined as pairs of circles of different size and at different distances. During the off-line phase, efficient local samplers are estimated by fitting solution paths in each of the local primitives, and the samplers' parameters are saved in a database. During the on-line phase, the problem at hand is decomposed into a set of subproblems, each corresponding to a local primitive. The parameters of the corresponding local samplers, which sample near relevant partial paths, are retrieved using the database. Finally, the local samplers synthesize a global sampler that considers the entire problem. The results are promising, achieving one to two orders of magnitude improvement over state-of-the-art methods.
CHECK THIS OUT:

\cite{kingston2021experience-foliations}

\cite{chamzas2021-learn-sampling}

\cite{moll2021hyperplan}

\cite{quintero-chamzas2021-motion-planning-in-the-dark}

\cite{pairet2021-path-planning-for-manipulation}

\cite{chamzas2020rep-learning}

\cite{chamzas2019using-local-experiences-for-global-motion-planning}

\cite{chamzas2022-motionbenchmaker}

\cite{chamzas2022-reconstruction}

%However, the need to efficiently store and compare these local primitives still exists. One idea is to construct efficient geometric abstractions of the local primitives on which recently developed large-scale hashing techniques~\cite{Anshu2014} could be used.  Targeting the problem with a local approach opens up more possible research areas, including: effectively finding the local distribution that is robust to changes; stitching together the local samplers and account for interference between them; and automatically detecting the local primitives that effectively decompose the global problem.
%
%I will be working with my advisors who collectively have expertise in robotics, computational geometry, motion planning, big data, and hashing. My work will be tested and applied to real robots: two UR5 arms and a Fetch robot. Also, I will explore storing my database in the cloud to enable robots to share experiences and improve collaboratively.
%
%\section{Broader Impacts}
%All developed software will be disseminated under an open source license within the Open Motion Planning Library, a popular robotics software maintained by our group. I am committed to sharing my science with the research world, and I will inform the academic community by publishing my results in interdisciplinary academic journals and international conferences. My work has the potential to bridge the motion planning community with the database and information retrieval communities, encouraging interdisciplinary collaboration and potentially transforming both fields. 
%
%Also, I plan to apply my research to the real world. More specifically, I will investigate the application of robotics in a hospital, which may impact the delivery of care. For example, a robot that could transport equipment, medication, and paperwork would free providers to concentrate on patients. I intend to build an interdisciplinary student team and foster collaboration with the proximal Texas Medical Center. Successfully developing an experienced-based framework in hospitals could serve as a pilot program to expand to other settings, such as oil rig inspections, space station maintenance, or building tasks, advancing safety and work performance in diverse environments.

\vspace{1em}
\footnotesize
\bibliography{papers.bib, extra.bib}
\end{document}
